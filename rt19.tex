% fix use of \sigma at the very end; give a new symbol like {\mathbb U}_{g}


\documentclass[11pt,notitlepage,english]{article}

\usepackage[numbers,sort&compress]{natbib}
\usepackage[hidelinks]{hyperref}
\usepackage{url}
\usepackage{fullpage}
\usepackage{amssymb}
\usepackage{amsbsy}
\usepackage{amsmath}
\usepackage[parfill]{parskip}
\usepackage{graphicx}
\usepackage{pgfplotstable}
\usepackage{tablefootnote}
\usepackage{caption}
\usepackage{soul}
\usepackage{authblk}
\usepackage{float}
\usepackage{placeins}
\usepackage{booktabs}

\hypersetup{
    colorlinks,
    linkcolor={red!50!black},
    citecolor={blue!50!black},
    urlcolor={blue!80!black}
}

\renewcommand{\familydefault}{\sfdefault}

\begin{document}
\captionsetup{format=hang}

\title{RareTarget: a machine-learning approach for learning molecular targets for rare disease therapies from Orphanet and RTX-KG2}

\setlength{\affilsep}{1em}

\author[1]{Amy~K.~Glen}
\author[2,3]{David~Koslicki}
\author[1,4]{Stephen~A.~Ramsey}
\affil[1]{School of Electrical Engineering and Computer Science\protect\\
  Oregon State University, Corvallis, Oregon USA\protect\\\mbox{ }}
\affil[2]{Computer Science and Engineering\protect\\
  Penn State University, State College, Pennsylvania USA\protect\\\mbox{ }}
\affil[3]{Huck Institutes of the Life Sciences\protect\\
  Penn State University, State College, Pennsylvania USA\protect\\\mbox{ }}
\affil[4]{Department of Biomedical Sciences\protect\\
  Oregon State University, Corvallis, Oregon USA\protect\\\mbox{ }}
 
\maketitle

\section{Preliminaries}
\subsection{General mathematical notation}\label{sec:math}
We'll denote the real numbers by ${\mathbb R}$, the nonnegative reals by
${\mathbb R}_{\geq 0}$, the negative reals by ${\mathbb R}_{-}$, the positive
integers by ${\mathbb N}$, and the set of positive integers less than or equal
to $n \in {\mathbb N}$ by ${\mathbb N}_n$.  Other than the reals and the natural
numbers, we'll denote sets by capital letters, like $S$.  For a discrete set
$S$, we'll denote its cardinality by $|S|$.  We'll denote a set difference by
using a left-leaning slash, like $A \setminus B$. We will denote maps by
lower-case Greek letters, like $\mu$, or by English words or abbreviations like
``logistic,'' ``log,'' ``Perm,'' or ``CEL''\@.  For any set $S$, we'll denote
its power set by ${\mathcal P}(S)$.  For any discrete set $S$, let a {\em
  permutation\/} of elements of $S$ be a bijection $\pi: S \rightarrow {\mathbb
  N}_{|S|}$, i.e., $\pi$ is an indexing function that defines a total order on
$S$.  We will denote the set of all permutations of elements of $S$ by
$\textrm{Perm}(S)$.  We'll denote scalar numeric values (real or integer) with
lower-case symbols, like $v$. Given a discrete set $S$, and $\pi \in
\textrm{Perm}(S)$, we'll denote real-valued (or nonnegative real-valued) vectors
corresponding to elements of $S$ ordered by $\pi$ as elements of ${\mathbb
  R}^{|S|}$ (or ${({\mathbb R}_{\geq 0})}^{|S|}$) using bolded lower-case
letters, like ${\boldsymbol z}$. We'll denote the length of a vector
$\boldsymbol{z}$ (i.e., the dimension of the vector space) by $\dim({\boldsymbol
  z})$.  For such a real-valued vector over the space of dimension equal to the
cardinality of $S$, it will be convenient to refer to the ``component of
$\boldsymbol{z}$ corresponding to element $s \in S$'', which will be understood
to mean the $\pi(s)$\textsuperscript{th} component of $\boldsymbol{z}$; we will
denote such a component by $[\boldsymbol{z}]_s$ or, for conciseness, $z_s$ (note
here, $s$ is an element of $S$ and not an index set element). We will denote a
map whose range is a Cartesian product space, by a boldfaced Greek letter, like
$\boldsymbol{\chi}$. To be clear, $\boldsymbol{\chi}$ is not a point in a
product space; rather, {\em its value for a given input\/} is a point in a
vector space (hence the use of boldface). We will denote real-valued matrices
over the space of dimension equal to the product of the cardinality of the
discrete set $S_1$ and the cardinality of the discrete set $S_2$, with bolded
upper-case letters, like ${\boldsymbol A}$. Assuming that the number of rows of
a real-valued matrix is $|S_1|$ and the number of columns of the matrix is
$|S_2|$, we will describe the matrix $\boldsymbol{A}$ as a ${\mathbb
  R}^{|S_1|\cdot |S_2|}$ matrix.  Given a permutation $\pi_1(S_1)$ providing a
correspondence between $S_1$ elements and ordering of rows, and a permutation
$\pi_2(S_2)$ providing a correspondence between $S_2$ elements and ordering of
columns, for any $s_1 \in S_1$ and $s_2 \in S_2$, we will denote the element of
$\boldsymbol{A}$ in the $\pi_1(s_1)$\textsuperscript{th} row and the
$\pi_2(s_2)$\textsuperscript{th} column by $A_{s_1,s_2}$. We will denote
real-valued tensors with bolded upper-case letters with a tilde, like
$\widetilde{\boldsymbol A}$. As with vectors and matrices, we will use set
element based notation to refer to tensor components; for example, if
$\widetilde{\boldsymbol A}$ is a ${\mathbb R}^{|S_1|\cdot |S_2| \cdot |S_3|}$
tensor, given permutations $\pi_1(S_1)$, $\pi_2(S_2)$, and $\pi_3(S_3)$, for any
elements $s_1 \in S_1$, $s_2 \in S_2$, and $s_3 \in S_3$, we'll denote the
$(\pi_1(s_1), \pi_2(s_2), \pi_3(s_3))$ component of $\widetilde{\boldsymbol A}$
by $A_{s_1,s_2,s_3}$. We'll denote the $L^2$~norm (i.e., Euclidean norm) of
vector ${\boldsymbol v}$ by $||{\boldsymbol v}||_2$ and the $L^1$~norm by
$||{\boldsymbol v}||_1$. We'll denote the open unit interval, the closed unit
interval, and the left half-open unit interval by $(0,1)$, $[0,1]$, and $(0,1]$,
  % chktex 9 respectively.  We'll abbreviate ``is defined by'' using the
  equivalence ($\equiv$) symbol.  For a set $S$, we'll denote the arithmetic
  mean of the values of a function $\mu: S \rightarrow {\mathbb R}$ over all
  elements of $S$ using angle brackets, $\langle \mu(s) \rangle_{s \in S}$. For
  any Cartesian product
\begin{equation}
  S \equiv S_1 \times \cdots \times S_n,
\end{equation}
we'll denote the projection map to the set $S_i$ by $\rho_{S_i}:S\rightarrow S_i$, where $\rho_{S_i}$ is defined by
\begin{equation}
\rho_{S_i}(s_1, \ldots, s_n) = s_i,
\end{equation}
where $(s_1, \ldots, s_n) \in S$. We'll denote a Cartesian product of elements
of a set of sets $\{S_i\}$ over an index set $I$, by a $\prod_{i \in I} S_i$.
For a Cartesian product $S_0 \times {( \prod_{i=1}^{l-1}{(S_i )})} \times S_l$
(where $l \in {\mathbb N}$), we'll define the case $l = 1$ to mean $S_0 \times
S_1$.  We'll denote the composition of two maps using the circle symbol $\circ$,
like $\nu \circ \epsilon$.  We'll denote the diagonal matrix whose elements are
the ordered elements of the vector ${\boldsymbol v}$ by
$\textrm{diag}({\boldsymbol v})$. We'll denote real-valued hyperparameters by
$u_1, u_2, u_3, \ldots$, and so on. We will denote parameter estimates or
predicted values using a hat (e.g., $\widehat{u}_1$ as an estimate for
hyperparameter $u_1$; and $\widehat{y}$ as a probabilistic prediction for label $y$).

\subsection{Subject-specific mathematical symbols}
 The key mathematical symbols for this method are summarized in
 Table~\ref{table:symboldefs} (see Appendix).  Let $G$ be the set of all human
 genes associated with disease-causing variants that are listed in the Orphanet
 database; we'll assume that each gene is identified by an Ensembl gene
 identifier, like {\em ENSG00000130158\/} (corresponding to the gene symbol {\em
   DOCK6\/}). Let $D$ denote the set of all Orphanet diseases, each represented
 by an Orphanet identifier like {\em ORPHA:974}.  Let $H$ denote the set of all
 symptoms appearing in Orphanet, each identified by a Human Phenotype Ontology
 (HPO) identifier like {\em HP:0002814}. Orphanet defines five symptom frequency
 categories, like {\em rare\/} or {\em occasional\/}; we will denote the set of
 frequency categories by $F$. Each Orphanet symptom frequency category
 corresponds to a percentage range of symptom frequencies, for example, {\em
   occasional\/} corresponds to the range $5--29$\%. We will simplify each
 frequency category's percentage range by taking the midpoint and dividing by
 100, to get a ``fractional midpoint'' summary of the frequency category
 (Table~\ref{table:freq}); for example, for the category {\em occasional}, the
 fractional midpoint frequency would be 0.17.
\begin{table}[!htb]
  \begin{center}
    \begin{tabular}{rrr} \toprule
      frequency category & \hspace{0.5cm}\%~of cases & \hspace{0.5cm}fractional midpoint \\\midrule\midrule
      always present & 100 & 1.00 \\\midrule
      very frequent & 80--99 & 0.90 \\\midrule
      frequent & 30--79 & 0.55 \\\midrule
      occasional & 5--29 & 0.17 \\\midrule
      rare & 1--4 & 0.02 \\\bottomrule
    \end{tabular}
    \caption{Symptom frequency categories as defined in the Orphanet database
      (represented by the set $F$), along with
      an approximate midpoint frequency for the frequency range (used to define
      values for $\xi$; see Eq.~\ref{eq:psi}).}\label{table:freq}
  \end{center}
\end{table}
Let $X \subset D \times H \times F$ denote a set of ordered triples of disease,
symptom, and frequency category appearing in Orphanet. Let $\xi: F \rightarrow
(0,1]$ % chktex 9 map each frequency category to the fractional midpoint of the
  category's symptom frequency range, as shown in Table~\ref{table:freq}; for
  example, $\xi(\textrm{\em occasional\/}) = 0.17$.  Let $\delta: G \rightarrow
  {\mathcal P}(D)$ map a gene $g$ to the set of all diseases $d \in D$ for which
  Orphanet annotates $g$ as being associated with disease $d$.  We will construct
  the complete set of gene-symptom associations $Q \subset G \times H$ from
  $\delta$ and $X$ by using $\delta$ to connect $g$ to a disease $d$ and $X$ to
  connect $d$ to $h$, for various diseases $d \in D$, as follows:
\begin{equation}
  Q \equiv \{(g,h) \in G\times H \; | \; \exists \, d \in \delta(g) \; \textrm{and} \;
    \exists \, f \in F 
\;    \textrm{sth.} \; (d,h,f) \in X\}.
\end{equation}
In practice, $Q$ is a sparse subset of $G \times H$, because the average number of symptoms per gene (XXX) is much smaller
then the total number, XXX, of symptoms in $H$.  We
assume that every gene $g \in G$ appears as the first component of at
least one element of $Q$. For $(g, h) \in Q$, let $\gamma:
Q \rightarrow {\mathcal P}(F)$ map $(g, h)$ to the set of
all frequency categories for occurrence of symptom $h$ for any disease that is
listed in Orphanet as being associated with gene $g$,
\begin{equation}
\gamma(g,h) \equiv \{ f \in F \; | \; \exists \, d \in \delta(g) \; \textrm{sth.} \; (d, h, f) \in X\}.
\end{equation}
Further, let
$\phi: G \times H \rightarrow (0,1]$ % chktex 9
map
$(g,h)$ to a ``maximum symptom frequency'' value for any disease caused by $g$ in
the case where $h$ is a symptom of a disease caused by $g$,
and to a global ``average symptom frequency'' in the case where
$h$ is not a symptom of any disease caused by $g$,
\begin{equation}
  \phi(g,h) \equiv
  \begin{cases}
    \underset{f \in \gamma(g, h)}{\max} \; \xi(f),
    & \textrm{if} \; (g,h) \in Q, \\
    \langle \phi(g',h') \rangle_{(g',h') \in Q}, & \textrm{if} \;
    (g,h) \not\in Q.
  \end{cases}
  \label{eq:psi}
\end{equation}

\subsection{Knowledge graph}

Let $R$ denote the set of all Biolink predicate types in the knowledge
graph, which is a labeled, directed multigraph $(N, E)$,
with node set $N$ and predicate-labeled edge set
$E \subset R \times N \times N$. Each node $n \in N$
denotes a concept, such as a specific gene, a specific biological pathway, a
specific biological process, or a specific symptom.  Let $C$ denote
the set of all Biolink node types (i.e., ``categories'') corresponding to any
node in $N$. Let $\kappa:N\rightarrow C$ map each
node to its Biolink category.  The concept node set $N$ should be
constructed such that (1)~a gene and the protein that it encodes, are identified
as the same ``gene'' node in the graph (with category {\em biolink:Gene\/}); and
(2)~concepts referring diseases, such as MONDO or ORPHA identifiers, are {\em
  not included in the graph's set of nodes}.  We will assume that $G
\subset N$ and $H \subset N$ (in practice, if
there is a gene $g$ or a symptom $h$ in Orphanet that is not in the knowledge
graph, we can drop the element from $G$ or $H$ and we can
drop all elements of $Q$ containing $g$ or $h$). Let
$\widetilde{\boldsymbol A} \in {\{ 0, 1 \}}^{|R| \cdot |N|
  \cdot |N|}$ denote the knowledge graph's adjacency tensor, which is
defined by
\begin{equation}
  A_{r,n,n'} \equiv \begin{cases}
    1 & \textrm{if} \; (r, n, n') \in E, \\
    0 & \textrm{if} \; (r, n, n') \not\in E.
    \end{cases}
\end{equation}
In this work, we are concerned with simple paths through the knowledge graph
that originate at a node from $G$ and that end at a symptom from $H$.  Thus,
we'll define a {\em path\/} in the knowledge graph to be an ordered pair
$({\boldsymbol n},{\boldsymbol r}) \in (G \times {( N^{\,l-1} )} \times H)
\times R^{\,l}$ of length $l$ hops, where $l \in {\mathbb N}$, ${\boldsymbol n}$
denotes the node sequence, and ${\boldsymbol r}$ denotes the sequence of
predictes (i.e., relationship types) for the edges connecting the nodes. It must
be the case that for all $i \in \{1, \ldots, l\}$, the path components satisfy
$(r_i, n_{i}, n_{i+1}) \in E$. Since we are looking to extract intermediate node
weights, in the formulation of our model for predicting $(g,h)$ pairs, we will
only consider paths whose lengths are in the range $\{2, \ldots, m\}$, where $m$
is a maximum path length that we will have to choose; in practice, we will
constrain paths to at most four hops by choosing $m=4$.

\section{Model definition and training}
Based on the gene-symptom associations $Q$ from Orphanet, the
symptom frequencies $\gamma$ (for gene-symptom pairs) from Orphanet, and the
knowledge graph ${\boldsymbol A}$, we aim to learn node weights and relationship type weights that
enable a path-based model to predict assocations between disease genes and the
diseases' associated symptoms.  More precisely, we desire to train a model that
for a gene $g \in G$ and symptom $h \in H$, will output an
``edge'' association score $z_{g,h} \in {\mathbb R}$ for the pair $(g, h)$.
(We will explain below in Section~\ref{sec:modelz} how
$z_{g,h}$ will actually be computed). We'll denote the probability corresponding
to $z_{g,h}$ (via the logistic function) by $\widehat{y}_{g,h} \in (0,1)$,
\begin{equation}
  \label{eq:defyhat}
  \widehat{y}_{g,h} \equiv \textrm{logistic}(z_{g,h}) = \frac{1}{1 + e^{-z_{g,h}}}.
\end{equation}
The value $z_{g,h}$ for a given pair $(g,h)$ will depend on various high-dimensional
model parameters and hyperparameters, which are not shown in the notation $z_{g,h}$,
but which will in practice be needed in order to compute a value $z_{g,h}$ for a given pair $(g,h)$.
We will discuss each of these types of parameters in turn, next.

\subsection{Model training set}
In order to train a model for computing $z_{g,h}$, we will need a contrastive set
of training examples.  By ``contrastive set'', we mean a
set $T \subset G\times H$ of gene-symptom pairs 
that include real gene-symptom associations (from $Q$, i.e.,
``positive'' examples) and gene-symptom pairs for which there is no disease for
which the gene is caually associated with the symptom (i.e., pairs not in
$Q$, or ``negative'' examples). It will be convenient to give a label
for the disjoint subsets of $T$ that are positive examples ($T_{+}$) and negative examples
($T_{-}$), such that
\begin{align}
  T_{+} &\equiv T \cap Q, \\
  T_{-} &\equiv T - T_{+}.
\end{align}
It is necessary that the positive examples in $T$ comprise only a
subset of $Q$, in order to be able to determine how well the model can
predict the ``class label'' (associated in Orphanet, or not) on gene-symptom
pairs that were not seen during training.  To represent the class label for each
gene pair $(g,h) \in T$, we'll define an indicator $y_{g,h}$ as
follows:
\begin{equation}
  y_{g,h} \equiv
  \begin{cases}
    1 & \textrm{if} \; (g,h) \in Q, \\
    0 & \textrm{if} \; (g,h) \not\in Q.
  \end{cases}
\end{equation}
We will also need a way to map between a gene $g$ and the set of all symptoms
that are paired with the gene (either through positive-class or negative-class examples)
in any element of $T$. Let $G_T \subset G$ be the set of
genes from any element of $T$, i.e.,
\begin{equation}
  G_T \equiv \{ \rho_G(t) \; | \; t \in T\},
\end{equation}
and similarly $G_{T_{+}}$ be the set of genes from any element of $T_{+}$.
Let $\sigma: G_T \rightarrow {\mathcal P}(H)$ map a gene $g$
to the set of all symptoms paired with $g$ in $T$, i.e.,
\begin{equation}
\sigma(g) \equiv \{ \rho_H(t) \; | \; t \in T \; \textrm{and} \; g = \rho_G(t) \}.
\end{equation}
Note that this includes both symptoms that are caused by disease gene $g$ {\em and\/}
symptoms that are not caused by disease gene $g$ (i.e., pairs $(g,h) \not\in
Q$).  For use in the prediction stage of the approach
(Section~\ref{sec:pred}), we will also need to define a function $\sigma_{+}:
G_{T_{+}} \rightarrow {\mathcal P}(H)$ that maps a gene $g$ to the
set of all symptoms caused by variants in $g$ (i.e., symptoms connected to $g$
via positive examples in $T$),
\begin{equation}
\sigma_{+}(g) \equiv \{ \rho_H(t) \; | \; t \in T_{+} \; \textrm{and} \; g = \rho_G(t)\}.
\end{equation}
Let $\epsilon:G_T \rightarrow {\mathcal P}(N)$ map
a gene $g$ to the set of all nodes traversed by simple paths\footnote{This means
that $g \not\in \epsilon(g)$ for any $g \in G_T$, since
simple paths by definition do not contain cycles.} (through the knowledge graph
$(N, E)$) of length at most $m$ starting from gene $g$ and
terminating on a symptom in $\sigma(g)$. Let the set of gene nodes within
the knowledge graph be denoted by $N_G$,
\begin{equation}
  N_G \equiv \{ n \in N \; | \; \kappa(n) = \textrm{\em biolink:Gene\/}\}.
\end{equation}
Let $\nu:{\mathcal P}(N)\rightarrow {\mathcal P}(N_G)$ map a set of nodes
to the subset that are genes, i.e., whose categories are {\em biolink:Gene},
\begin{equation}
  \nu(S \subset N) \equiv S \cap N_G.
\end{equation}
We'll define the composition map $\alpha \equiv (\nu \circ \epsilon): G_T \rightarrow {\mathcal P}(N_G)$, which maps a disease
gene $g$ to the set of intermediate nodes of type gene that are traversed by
paths up to length $m$ from $g$ to associated symptoms.  Let $C_T$ denote the set of categories of nodes that are reachable by
paths of up to length $m$ with pairs $(g,h)$ of starting and ending nodes that
are contained in $T$ ({\em excluding\/} the category {\em biolink:Gene\/}),
\begin{equation}
  C_T \equiv \{ \kappa(n) \; | \; \exists \, g \in G_T
  \;\, \textrm{sth} \;\, n \in \epsilon(g) \} \setminus \{ \textrm{\em biolink:Gene\/} \}.
\end{equation}

\subsection{Node category weights $\boldsymbol{k}$}
Based on an ordering selected arbitrarily from $\textrm{Perm}(C_T)$,
let the vector $\boldsymbol{k} \in {\mathbb R}_{\geq  0}^{|C_T|}$
represent node category weights, which will be universal for our model (i.e.,
this set will not be learned separately for each disease gene $g$). We'll denote
by $k_c$ the component of $\boldsymbol{k}$ corresponding to category $c$.

\subsection{The gene node weights ${\boldsymbol t}^{(g)}$}
Given a gene $g \in G_T$, based on a fixed ordering selected
arbitrarily from $\textrm{Perm}(\alpha(g))$, let the vector ${\boldsymbol
  t}^{(g)} \in {({\mathbb R}_{\geq 0})}^{|\alpha(g)|}$ represent node weights
for intermediate nodes of type gene for gene-to-symptom paths through the
knowledge graph that start at the gene $g$.  We'll denote the component of
$\boldsymbol{t}^{(g)}$ corresponding to node $n \in \alpha(g)$ as ${t^{(g)}}_n$.

\subsection{The node weights}
For each gene $g \in G_T$, we'll arbitrarily pick a fixed ordering $\pi^{(g)} \in
\textrm{Perm}(\epsilon(g))$. Also for each such gene $g$, let
us define
$\boldsymbol{\chi}:
       {({\mathbb R}_{\geq 0})}^{|\alpha(g)|} \times {({\mathbb R}_{\geq  0})}^{|C_T|}
        \rightarrow {({\mathbb R}_{\geq 0})}^{|\epsilon(g)|}$ such that for any
$n \in \epsilon(g)$, 
\begin{equation}
  {[\boldsymbol{\chi}({\boldsymbol t}^{(g)}, {\boldsymbol k})]}_n
   \equiv
  \begin{cases}
    {t^{(g)}}_n & \textrm{if} \;\, n \in \alpha(g), \\
    k_{\kappa(n)} & \textrm{otherwise},
  \end{cases}
\end{equation}
where $\pi^{(g)} \in \textrm{Perm}(\epsilon(g))$ is the ordering of the
dimensions of the range of $\boldsymbol{\chi}$.

\subsection{The relationship type or ``predicate'' weight $({\boldsymbol w})$ parameter vector}
Let ${\boldsymbol w} \in {({\mathbb R}_{\geq 0})}^{|R|}$ be a vector set of
relationship type (a.k.a.\ ``predicate'') weights.  Let
${\boldsymbol A}^{({\boldsymbol w})} \in {{\mathbb R}_{\ge 0}}^{|N| \cdot |N|}$ denote the 
weighted adjacency matrix of the graph in which each edge has been weighted
by its predicate according to the predicate weight vector ${\boldsymbol w}$,
calculated as an inner product,
\begin{equation}
{\boldsymbol A}^{({\boldsymbol w})} \equiv {\boldsymbol w} \cdot \widetilde{\boldsymbol A},
\end{equation}
or equivalently, with an explicit summation,
\begin{equation}
{[{\boldsymbol A}^{({\boldsymbol w})}]}_{n,n'} \equiv \sum_{r \in R} w_r \; {[\widetilde{\boldsymbol A}]}_{r,n,n'}.
\end{equation}
The predicate weight vector ${\boldsymbol w}$ is one of the key outputs of the
model training that we will carry out based on Orphanet data, in Section~\ref{sec:learnwf}. For any $g \in G_T$ and
${\boldsymbol w} \in {({\mathbb R}_{\geq 0})}^{|R|}$,
let ${\boldsymbol A}^{({\boldsymbol w}, g)}$ denote a
  ${({\mathbb R}_{\geq 0})}^{|\epsilon(g)|\cdot|\epsilon(g)|}$ matrix whose row and
column index orderings are $\pi^{(g)}$, and whose components satisfy:
\begin{equation}
  {[{\boldsymbol A}^{({\boldsymbol w},g)}]}_{n,n'} \equiv
  \begin{cases}
    {[{\boldsymbol A}^{({\boldsymbol w})}]}_{n,n'} & \textrm{if} \;\, n \in \epsilon(g) \;\,
    \textrm{and} \;\, n' \in \epsilon(g), \\
      0 & \textrm{otherwise}.
  \end{cases}
\end{equation}
  In other words, ${\boldsymbol A}^{({\boldsymbol w},g)}$ is a projection of
  ${{\boldsymbol A}^{({\boldsymbol w})}}$ to the square submatrix whose rows
  and columns correspond to elements of $\epsilon(g)$ with ordering $\pi^{(g)}$.

\subsection{Modeling the real-valued $(g,h)$ edge associations $z_{g,h}$}\label{sec:modelz}
Given $g \in G_T$, ${\boldsymbol k}$, ${\boldsymbol w}$,
${\boldsymbol t}^{(g)}$, and
${\boldsymbol A}^{({\boldsymbol w}, g)}$, we will compute the $(g,h)$ edge association score
$z_{g,h} \in {\mathbb R}$ as the value of a function
$\zeta_{g,h}:
{({\mathbb R}_{\geq 0})}^{|\alpha(g)|} \times
{({\mathbb R}_{\geq 0})}^{|C_T|} \times
{({\mathbb R}_{\geq 0})}^{|R|} \times
{\mathbb R}_{-} \rightarrow {\mathbb R}$,
\begin{align}
  z_{g,h} &\equiv \zeta_{g,h}({\boldsymbol t}^{(g)}, {\boldsymbol k}, {\boldsymbol w},
   f), \label{eq:defzgs} \\
 \zeta_{g,h}({\boldsymbol t}^{(g)}, {\boldsymbol k}, {\boldsymbol w},
      f)   & \equiv
      \sum\limits_{l = 2}^{m} {\left[
        {\boldsymbol A}^{({\boldsymbol w},g)}
          {\left(\textrm{diag}(
            \boldsymbol{\chi}({\boldsymbol t}^{(g)}, {\boldsymbol k})
            ) \, {\boldsymbol A}^{({\boldsymbol w},g)}
      \right)}^{l-1}\right]}_{h,g} + f, 
  \label{eq:master}
\end{align}
 where $f \in {\mathbb R}_{-}$ is a scalar offset that corresponds to the
 baseline log-odds of an association (via some disease in Orphanet) between $g$
 and $h$ if there are no paths between them in the knowledge graph. In our
 approach, the value of $f$ will be learned as a model parameter.  In
 Eq.~\ref{eq:master}, the form ${(\textrm{diag}(\boldsymbol{\chi}({\boldsymbol
     t}^{(g)}, {\boldsymbol k}) ) \, {\boldsymbol A}^{({\boldsymbol w},g)}
   )}^{l-1}$ denotes $l-1$ alternating matrix multiplications of the diagonal
 node weight matrix $\textrm{diag}(\boldsymbol{\chi}({\boldsymbol t}^{(g)},
 {\boldsymbol k}))$ and the edge-weighted adjacency matrix ${\boldsymbol
   A}^{({\boldsymbol w},g)}$.  That is,
\begin{equation}
{\left(\textrm{diag}(\boldsymbol{\chi}({\boldsymbol t}^{(g)}, {\boldsymbol k}) )
  \; {\boldsymbol A}^{({\boldsymbol w},g)} \right)}^{l-1} = \,
\textrm{diag}(\boldsymbol{\chi}({\boldsymbol t}^{(g)}, {\boldsymbol k})) \;
  {\boldsymbol A}^{({\boldsymbol w},g)} \;
  \textrm{diag}(\boldsymbol{\chi}({\boldsymbol t}^{(g)}, {\boldsymbol k}))
    \, {\boldsymbol A}^{({\boldsymbol w},g)} \cdots,
\end{equation}
with $l-1$ total factors (including $l-1$ instances of
$\textrm{diag}(\boldsymbol{\chi})$ and $l-1$ instances of ${\boldsymbol
  A}^{({\boldsymbol w},g)}$), and where multiplication proceeds from right to
left.
Let us define a function
$\upsilon_{g,h}: {({\mathbb R}_{\geq 0})}^{|\alpha(g)|} \times {({\mathbb R}_{\geq 0})}^{|C_T|} \times {({\mathbb R}_{\geq 0})}^{|R|} \times {\mathbb R}_{-} \rightarrow (0,1)$,
as follows:
\begin{equation}\label{eq:upsilon}
\upsilon_{g,h} \equiv \textrm{logistic} \circ \zeta_{g,h}.
\end{equation}
We can then model the probability predictions
by combining
Eq.~\ref{eq:defyhat},
Eq.~\ref{eq:defzgs},
Eq.~\ref{eq:master}, and
Eq.~\ref{eq:upsilon}, 
\begin{align}
  \widehat{y}_{g,h} & \equiv \upsilon_{g,h}({\boldsymbol t}^{(g)}, {\boldsymbol k}, {\boldsymbol w},
  f) \\
  &=
\frac{1}{1 + e^{-\zeta_{g,h}({\boldsymbol t}^{(g)}, {\boldsymbol k}, {\boldsymbol w},
  f)}}.
\end{align}
Let $\psi: G\rightarrow {\mathbb R}_{\geq 0}$ map a gene $g \in G$ to
a per-gene weight normalization value that we define by
\begin{equation}
\psi(g) \equiv \sum_{h \in \sigma(g)} \phi(g,h).
\end{equation}

\subsection{The target-finding model's causal gene-specific objective function}
Let $\textrm{CEL}: \{0,1\} \times (0,1) \rightarrow {\mathbb R}_{\geq 0}$ denote the cross-entropy loss
between a binary class label $y \in \{0,1\}$ and a continuous-valued prediction
$\widehat{y} \in (0,1)$,
\begin{equation}
\textrm{CEL}(y, \, \widehat{y}) \equiv -y \log(\widehat{y}) - (1-y) \log(1-\widehat{y}).
\end{equation}
For a given $g \in G_T$, we will construct a gene-specific objective
function
$\lambda^{(g)}: {({\mathbb R}_{\geq 0})}^{|\alpha(g)|} \times {({\mathbb R}_{\geq 0})}^{|C_T|} \times {({\mathbb R}_{\geq 0})}^{|R|} \times {\mathbb R}_{-} \rightarrow {\mathbb R}_{\geq 0}$
as the sum of a weighted average prediction loss (over symptoms
in $\sigma(g)$) and an elastic net regularization of ${\boldsymbol t}^{(g)}$
with respective $L^1$ and $L^2$ penalty coefficients $u_1 \in {\mathbb R}_{\geq
  0}$ and $u_2 \in {\mathbb R}_{\geq 0}$,
\begin{equation}
  \label{eq:lambdag}
  \lambda^{(g)}(
{\boldsymbol t}^{(g)}, {\boldsymbol k}, {\boldsymbol w},
f) \equiv  \frac{1}{\psi(g)}
  \sum\limits_{h \in \sigma(g)}
  \left[ \phi(g, h)\, \textrm{CEL}\!\left(y_{g,h}, \, \widehat{y}_{g,h} \right) \right]  +  
  \frac{u_1}{|\alpha(g)|} ||{\boldsymbol
    t}^{(g)}||_1
  + \frac{u_2}{|\alpha(g)|} {\bigl(||{\boldsymbol t}^{(g)}||_2\bigr)}^2.
\end{equation}
Using Eq.~\ref{eq:upsilon}, we can express the above as
\begin{multline}
  \label{eq:lambdag}
  \lambda^{(g)}(
{\boldsymbol t}^{(g)}, {\boldsymbol k}, {\boldsymbol w},
f) \equiv  \frac{1}{\psi(g)}
  \sum\limits_{h \in \sigma(g)}
  \left[ \phi(g, h)\, \textrm{CEL}\!\left(y_{g,h}, \, \upsilon_{g,h}(
{\boldsymbol t}^{(g)}, {\boldsymbol k}, {\boldsymbol w},
f)\right) \right]  + \\
  \frac{u_1}{|\alpha(g)|} ||{\boldsymbol
    t}^{(g)}||_1
  + \frac{u_2}{|\alpha(g)|} {\bigl(||{\boldsymbol t}^{(g)}||_2\bigr)}^2.
\end{multline}

\subsection{The target-finding models's overall (across causal genes) obj.~function}

We'll denote the concatenation of all of the components of all of
the vectors ${\boldsymbol t}^{(g)}$ over all $g \in G_T$, by
$\boldsymbol{t} \in {({\mathbb R}_{\geq 0})}^{\sum_{g \in G_T} |\alpha(g)|}$. Note that
\begin{equation}
\dim({\boldsymbol t}) = \sum_{g \in G_T} |\alpha(g)|.
\end{equation}
Then our overall symptom-frequency-weighted objective function
$\lambda: {({\mathbb R}_{\geq 0})}^{\dim({\boldsymbol t})} \times {({\mathbb R}_{\geq 0})}^{|C_T|} \times {({\mathbb R}_{\geq   0})}^{|R|} \times {\mathbb R}_{-} \rightarrow {\mathbb R}_{\geq 0}$
 is defined by an average of
 $\lambda^{(g)}$ over $g \in G_T$, plus a squared $L^2$ regularization
 of weight vector ${\boldsymbol k}$ with coefficient $u_3 \in {\mathbb R}_{\geq 0}$
 plus a squared $L^2$ regularization of
weight vector ${\boldsymbol w}$ with coefficient $u_4 \in {\mathbb R}_{\geq 0}$,
\begin{equation}
 \lambda({\boldsymbol t}, {\boldsymbol k}, {\boldsymbol w}, f) \equiv \left[\frac{1}{|G_T|}
  \sum\limits_{g \in G_T} \lambda^{(g)}({\boldsymbol t}^{(g)}, {\boldsymbol k}, {\boldsymbol w}, f)\right] +  
  \frac{u_3}{|C_T|} {(||{\boldsymbol k}||_2)}^2 +
  \frac{u_4}{|R|} {(||{\boldsymbol w}||_2)}^2,
\end{equation}
which expands (using the definition in Eq.~\ref{eq:lambdag}) to,
\begin{multline}
  \lambda({\boldsymbol t}, {\boldsymbol k}, {\boldsymbol w}, f) = \frac{1}{|G_T|}
  \sum\limits_{g \in G_T}  \Biggl\{ \frac{1}{\psi(g)}
  \sum\limits_{h \in \sigma(g)}
  \left[ \phi(g, h)\, \textrm{CEL}\!\left(y_{g,h}, \; \upsilon_{g,h}({\boldsymbol
      t}^{(g)},{\boldsymbol k},{\boldsymbol w},f)\right) \right] + \\
  \frac{u_1}{|\alpha(g)|} ||{\boldsymbol
    t}^{(g)}||_1
  + \frac{u_2}{|\alpha(g)|} {\bigl(||{\boldsymbol t}^{(g)}||_2\bigr)}^2 \Biggr\} +
  \frac{u_3}{|C_T|} {(||{\boldsymbol k}||_2)}^2 +
  \frac{u_4}{|R|} {(||{\boldsymbol w}||_2)}^2.
\end{multline}

\subsection{For fixed $(u_1,u_2,u_3,u_4)$ hyperparameters, find $(\widehat{\boldsymbol k}, \widehat{\boldsymbol w}, \widehat{f})$}\label{sec:learnwf}
We will very likely want to dedicate a subsect of gene-disease associations (the
``training set''), to the task of learning $({\boldsymbol t}, {\boldsymbol k}, {\boldsymbol w}, f)$.
For fixed hyperparameters $(u_1, u_2, u_3, u_4)$, we can try to
estimate the model parameters $({\boldsymbol t}, {\boldsymbol k}, {\boldsymbol w}, f)$ by
minimizing the overall loss function,
\begin{equation}
  (\widehat{\boldsymbol t}, \widehat{\boldsymbol k}, \widehat{\boldsymbol w}, \widehat{f}) = \underset{({\boldsymbol
      t},{\boldsymbol k},{\boldsymbol w},f)}{\textrm{argmin}} \; \lambda({\boldsymbol t}, {\boldsymbol k},{\boldsymbol w}, f),
  \label{eq:optqw}
\end{equation}
probably with high node weight penalization coefficients $u_1$ and $u_2$ since
${\boldsymbol t}$ is $\sum_g |\alpha(g)|$-dimensional (i.e.,
extremely high-dimensional). The goal at this stage is to get good estimates
$\widehat{\boldsymbol k}$, $\widehat{\boldsymbol w}$ and $\widehat{f}$, and the likely sparse resulting estimate
$\widehat{\boldsymbol t}$ on the training set would be discarded.

\section{Given $(\widehat{\boldsymbol k}, \widehat{\boldsymbol w}, \widehat{f})$, predict targets for a specific causal gene $g$}\label{sec:pred}
With $\widehat{\boldsymbol w}$ and $\widehat{f}$ estimated, the model can (hopefully) be
used for prediction for targets for a rare disease caused by a specific gene
$g \in G$. Here, ``prediction'' amounts to scoring each possible intermediate
node $n \in \alpha(g)$ for its relevance as a mediator between $g$ and all the
symptoms $\sigma_{+}(g) \subset H$. More formally, for any $g \in
G$ such that $(g,h) \not\in T$ for any $h \in H$,
and given symptoms $\sigma_{+}(g)$ for gene $g$, and a set of symptoms that we would
randomly sample that are {\em not\/}
associated with the rare disease caused by $g$ (together denoted by the set $\sigma(g)$),
and given that $y_{g,h}$ is defined as one for $h \in \sigma_{+}(g)$ and 0 otherwise, 
we could predict our intervention
targets by looking for high-scoring components of the (length $|\alpha(g)|$)
vector $\widehat{\boldsymbol t}^{(g)}$; we would estimate that vector by:
\begin{equation}
  \widehat{\boldsymbol t}^{(g)} = \underset{{\boldsymbol t}^{(g)}}{\textrm{argmin}}
  \biggl[
    \lambda^{(g)}\bigl({\boldsymbol t}^{(g)},\widehat{\boldsymbol k},\widehat{\boldsymbol w},\widehat{f}\bigr)
    \biggr].
\end{equation}
We can obtain the equation for predicting the nodes that are likely mediators of causal
paths between causal gene $g$ and symptoms $\sigma(g)$ by using Eq.~\ref{eq:lambdag}, 
\begin{multline}
  \label{eq:nodeweights}
  \lambda^{(g)}({\boldsymbol t}^{(g)}, \widehat{\boldsymbol k},
  \widehat{\boldsymbol w}, \widehat{f}) =  \frac{1}{\psi(g)}
  \sum\limits_{h \in \sigma(g)} \left[
   \phi(g, h)\, \textrm{CEL}\!\left(y_{g,h}, \upsilon_{g,h}({\boldsymbol
     t}^{(g)},\widehat{\boldsymbol k}, \widehat{\boldsymbol w},\widehat{f})\right) \right] + \\
  \frac{u_1}{|\alpha(g)|}
  ||{\boldsymbol t}^{(g)}||_1
  + \frac{u_2}{|\alpha(g)|} {(||{\boldsymbol t}^{(g)}||_2)}^2.
\end{multline}
Finding $\textrm{argmin}_{{\boldsymbol t}^{(g)}} (\lambda^{(g)})$ will involve computing the gradient component
\begin{equation}
  \frac{\partial \lambda^{(g)}}{\partial t^{(g)}_n}({\boldsymbol t}^{(g)}, \widehat{\boldsymbol k}, \widehat{\boldsymbol w}, \widehat{f}),
\end{equation}
which involves computing (for each $l$) the gradient components (for all $n \in \alpha(g)$),
\begin{equation}
  \frac{\partial}{\partial t^{(g)}_n}{\left[
        {\boldsymbol A}^{({\boldsymbol w},g)}
      {\left(\textrm{diag}(\boldsymbol{\chi}({\boldsymbol t}^{(g)},{\boldsymbol k})) \,
        {\boldsymbol A}^{({\boldsymbol w},g)}
     \right)}^{l-1}\right]}_{h,g}.
\end{equation}
That gradient calculation may have to be done numerically unless we can find an analytic approach.

\section{Possible extensions}
There are a couple of ways the above-described method might be extended:
\begin{enumerate}
  \item Maybe there is also a way to work in a generic ``prior'' on ${\boldsymbol t}$,
    by node category type, using Bayesian methods. Have to think about how that could be done, though.
  \item We might need incorporate a path-length-specific weights,
    ${\boldsymbol h} \in {\mathbb R}^{(m-1)}$, into Eq.~\ref{eq:master}.
\end{enumerate}

\section{Questions to explore}
\begin{enumerate}
\item If we partition the set of symptoms $\sigma_{+}(g)$ that are associated with gene $g$ into two sets $S_1$ and $S_2$
  of equal size, and if we calculate (using Eq.~\ref{eq:nodeweights}) two different node weight vectors
  ${\boldsymbol t}^{(g)}_1$ and ${\boldsymbol t}^{(g)}_2$ from $S_1$ and $S_2$, we could measure the correlation
  $\textrm{cor}({\boldsymbol t}^{(g)}_1, {\boldsymbol t}^{(g)}_2)$ and test its significance vs.\ a null model.
\end{enumerate}

\appendix

\section{Supplementary Information}
\begin{table}[!htb]
  \begin{center}
      {\tiny 
    \begin{tabular}{cll}\toprule
      symbol & definition in plain English & example or symbolic definition \\\midrule\midrule
      $G$ & all genes appearing in Orphanet & {\em ENSG00000130158\/} \\\midrule
      $D$ & all diseases appearing in Orphanet & {\em ORPHA:974\/} \\\midrule
      $H$ & all symptoms appearing in Orphanet & {\em HP:0002814\/} \\\midrule
      $F$ & all Orphanet symptom frequency categories & {\em occasional\/} \\\midrule
      $Q$ & all gene-symptom associations (``links'') in Orphanet & $(\textrm{\em \/ENSG00000102967}, \textrm{\em \/HP:0000378})$ \\\midrule
      $T$ & training set $T \subset G\times H$ &  $(\textrm{\em \/ENSG00000102967}, \textrm{\em \/HP:0000378})$ \\\midrule
      $T_{+}$ & training set of positive examples & $T \cap L$  \\\midrule
      $T_{-}$ & training set of negative examples & $T - T_{+}$  \\\midrule
      $G_T$ & genes appearing in the training set $T$ &  {\em ENSG00000102967\/} \\\midrule
      $N$ & concept nodes in the knowledge graph &  {\em ENSG00000130158}, {\em GO:0032303}, {\em R-HSA-211979\/} \\\midrule
      $E$ & predicate-labeled edges in the knowledge graph &
      $(\textrm{\em \/ENS00000102967}, \textrm{\em \/HP:0003376}, \textrm{\em \/biolink:affects})$ \\\midrule
      $R$ & Biolink predicate types in the knowledge graph & {\em biolink:causes\/} \\\midrule
      $C$ & Biolink node types (``categories'') in the knowledge graph & {\em biolink:Gene\/} \\\midrule
      $C_T$ & categories of intermediate nodes in the graph & {\em biolink:Gene}, {\em biolink:Pathway}, {\em biolink:BiologicalProcess\/} \\\midrule
      $M$ & ``mediator'' nodes traversed by $g \rightarrow \cdots \rightarrow s$ paths & \\\midrule
      $N_G$ & all gene nodes in the knowledge graph & {\em ENSG00000130158\/} \\\midrule
      $\widehat{y}_{g,h}$ & predicted probability that $(g,h) \in L$ & $\widehat{y}_{g,h} \in (0,1)$ \\\midrule
      $\sigma$ & maps gene $g$ to the set of all associated symptoms in $T$ & \\\midrule
      $\sigma_{+}$ & maps gene $g$ to the set of all associated symptoms in $T_{+}$ & \\\midrule
      $\kappa$ & maps a node $n$ to the node's Biolink category &
      $\kappa(\textrm{\em ENSG00000130158\/}) = \textrm{\em biolink:Gene\/}$ \\\midrule
      $\rho_S$ & projection map to the $S$ component of a tuple & $\rho_{S_i}(s_1, \ldots, s_n) = s_i$ \\\midrule
      $\gamma$ & maps gene-symptom pair to frequencies & $\gamma(\textrm{\em \/ENSG00000001626}, \textrm{\em \/HP:0002783}) \rightarrow \{\textrm{\em \/occasional}, \textrm{\em \/frequent}\}$ \\\midrule
      $\phi$ &map gene-symptom pair to max frequency & $\phi(\textrm{\em \/ENSG00000001626}, \textrm{\em \/HP:0002783}) \rightarrow \textrm{\em \/frequent}$ \\\midrule
      $\psi$ & maps a gene to the sum of $\phi(g,h)$ over all $h \in \sigma(g)$ & $\psi(g) \in {\mathbb R}_{\geq 0}$ \\\midrule
      $\nu$ & maps a set of nodes to the subset of nodes that are genes &
      $\nu(\{\textrm{ENSG00000130158}, \textrm{HP:0002814}\}) = \{\textrm{ENSG00000130158}\}$ \\\midrule
      $\alpha$ & maps a gene to the set of gene nodes reachable from $g$ &
      \\\midrule
      ${\boldsymbol t}^{(g)}$ & the node weights for nodes reachable from $g$ &
      ${\boldsymbol t}^{(g)} \in {({\mathbb R}_{\geq 0})}^{|\alpha(g)|}$ \\\midrule
      ${\boldsymbol k}$ & the weights for node category types & ${\boldsymbol k} \in {({\mathbb R}_{\geq 0})}^{|C_T|}$ \\\midrule
      ${\boldsymbol w}$ & the weights for edge predicate types & ${\boldsymbol w} \in {({\mathbb R}_{\geq 0})}^{|R|}$ \\\midrule
      $f$ & the universal prediction score offset & $f \in {\mathbb R}_{-}$ \\\midrule
      $\upsilon_{g,h}$ & maps a parameter tuple to a prediction score & $\upsilon({\boldsymbol t}^{(g)}, {\boldsymbol k},{\boldsymbol w}, f)$ \\\midrule
      $m$ & the maximum number of edge hops for paths & $m=4$ \\\midrule
      $u_1$ & hyperparameter for $L^1$ norm penalty for ${\boldsymbol t}^{(g)}$ & $u_1 \in {\mathbb R}_{\geq 0}$ \\\midrule
      $u_2$ & hyperparameter for $L^2$ norm penalty for ${\boldsymbol t}^{(g)}$ & $u_1 \in {\mathbb R}_{\geq 0}$ \\\midrule
    \end{tabular}
    }
  \end{center}
  \caption{Table of key symbol definitions}\label{table:symboldefs}
\end{table}

\end{document} 
